\section{Gyrator como simulador de un inductor}

El gyrator es un circuito cuya impedancia de entrada puede aproximarse a la de una bobina con resistencia en serie en cierto rango de frecuencias. Se muestra un ejemplo en la figura \ref{fig:ej2_gyrator_simulacion_inductor}. Esta es la implementaci\'on que se usa en los filtros.

Adem\'as del rango acotado de funcionamiento, la simulaci\'on del inductor tiene otras limitaciones, de las cuales se mencionan dos ejemplos a continuaci\'on:

\begin{description}	
	\item[Alamcenamiento energ\'etico] no puede almacenar energ\'ia de la misma manera que un inductor. La magnitud de la fem producida ante cambios de corriente $\left( V = \frac{di}{dt} \right) $ tiene limitaciones propias de las caracter\'isticas el\'ectricas del circuito (ej.: el slew-rate del op-amp limita la velocidad de cambio de la tensi\'on) 
	
	\item[Efecto mutua inducci\'on] No crean campos magn\'eticos de la misma forma que los inductores, por lo que no se puede conseguir un efecto de mutua inducci\'on entre un gyrator y un inductor o entre dos gyrators de la misma forma que s\'i sucede entre dos inductores. 

\end{description}





\begin{figure*}[htbp]
	\centering
	\begin{circuitikz}[scale=1]
%\draw[help lines]   			grid(6,6);	
\draw								(0,3)
		to [R=$R_L$,i>^=$i_{in}$]	(3,3)
		to [L=$L{=}R_LRC$]			(3,0) node[ground]{};
\end{circuitikz}

	\begin{circuitikz}[scale=1]
\draw[help lines]   			grid(6,6);	

\draw									(0,3)
		to [short, o-*, i>^=$i_{in}$]	(1,3)
		to [short, i>_=$i_A$]			(1,4)
		to [R=$R_L$]				(3,4)
		to [short] 					(3,5)
		to [short]					(5.5,5)
		to [short]					(5.5,3.5)

									(3,2)
		to [short]					(3,3)
									(1,3)
		to [short, i>^=$i_B$]		(1,2)
		to [C=$C_1$]				(3,2)
		to [R=$R_1$]				(3,0) node[ground]{}
		
									(4.25,3.5)
		node[op amp](opamp){};
\end{circuitikz}

	\caption{Uso de gyrator como inductor}
	\label{fig:ej2_gyrator_simulacion_inductor}
\end{figure*}




En la secci\'on \ref{ssec:ej2_obtencion_zin_gyr} se obtiene que la impedancia de entrada del gyrator es

\begin{equation}
	Z_{in}= \frac{ sC_{gyr}R_{gyr}R_L+R_L}{sC_{gyr}(R_{gyr}(1-K)+R_L) + 1} \label{eq:ej2_zin_gyr_sin_aprox} \\
\end{equation}
	en donde	$K  \approx\frac{1}{1+\frac{s}{2\pi BWP}}$ (ecuaci\'on \ref{eq:ej2_K}).
 
Si el denominador puede aproximarse a 1 (ver secci\'on \ref{ssec:ej2_zin_gyr_aprox_inductor}), la impedancia de entrada se puede aproximar a la del modelo de un inductor con resistencia serie con valores $L=CR_LR_1$ y $R_{coil}=R_L$

\begin{align}
	Z_{in} &= sC_{gyr}R_LR_{gyr}+R_L \\
 	\abs{Z_{in}} &= R_L \, \sqrt{4\pi^2f^2C_{gyr}^2R_{gyr}^2+1} \\
 	\phase{Z_{in}} &= arctg(2\pi fC_{gyr}R_{gyr})
	\label{eq:ej2_zin_gyrator_con_aprox}
\end{align}



\subsection{Limitaciones en frecuencia} \label{ssec:ej2_zin_gyr_aprox_inductor}

%condiciones para la aproximacion
Se analiza el denominador de la impedancia de entrada (ecuaci\'on \ref{eq:ej2_zin_gyr_sin_aprox}) y se busca en qu\'e rango de frecuencias puede aproximarse a 1. Siendo $D(s)$ el denominador:

\[D(s) = sC_{gyr}(R_{gyr}(1-\frac{1}{1+\frac{s}{2\pi BWP}})+R_L) + 1 \]
\[	= \frac{C_{gyr}R_T\, s^2 + 2\pi BWP\, C_{gyr} R_L \, s + 2\pi BWP}{s+2\pi BWP}   \]
Siendo $R_T = R_L + R_{gyr}$.
\[=\frac{\frac{C_{gyr}R_T}{2\pi BWP}s^2 + C_{gyr}R_L s +1}{\frac{s}{2\pi BWP}+1}\]

A continuaci\'on se analizan los polos y ceros de $D(s)$:

\begin{description}

	\item[\textbf{Cero doble en $f_{cero} = \frac{1}{2\pi}\sqrt{\frac{2\pi BWP}{C_{gyr}R_T}}$: }]



Se observa que el denominador tiene un cero doble en $f_0 = \frac{1}{2\pi}\sqrt{\frac{2\pi BWP}{C_{gyr}R_T}}$. La existencia o no del sobrepico (y de existir, su altura y posici\'on) depende de la relaci\'on entre el coeficiente lineal $b = C_{gyr}R_L$ y $f_0$. 

\[b = \frac{2\xi}{2\pi f_0}\]
\[\xi = \frac{2\pi f_0 b}{2}\]
\[\xi = \frac{\sqrt{\frac{2\pi BWP}{C_{gyr}R_T}} \cdot C_{gyr}R_L}{2}   \]
\[ = \frac{1}{2}R_L\sqrt{\frac{2\pi BWP C_{gyr}}{R_L + R_{gyr}}}\]
Si $R_{gyr} \gg R_L$:
\[ \approx \frac{1}{2}R_L\sqrt{\frac{2\pi BWP C_{gyr}}{R_{gyr}}}\]
Si $\xi \leqslant \frac{1}{\sqrt{2}}$, $D(s)$ tiene sobrepico (va a ser cada vez mas pronunciado mientras m\'as chico sea $\xi$). 


	\item[\textbf{Polo simple en $f_{polo} = BWP$}]

\end{description}

$f_{cero} < f_{polo} \Rightarrow$ es el cero de $D(s)$ el que delimita cuando el gyrator deja de comportarse como inductor.




\subsection{Eleccion de componentes}

Sabiendo que $R_L = 50\Omega$, $L = 0.02H$ y $L = R_L R_{gyr} C_{gyr}$, se obtienen siguientes valores:
\begin{align*}
	R_{gyr} &= 39k\Omega \\
	C_{gyr} &= 10nF
\end{align*}


Analizando la $f_{cero}$  se obtiene una aproximaci\'on de cu\'ando la impedancia de entrada del gyrator deja de ser aproximadamente la de un inductor:

\[f_{cero} = \frac{1}{2\pi}\sqrt{\frac{2\pi 6MHz}{10nF39K\Omega}} \approx 35.0KHz\]

Esto coincide con lo observado en la figura \ref{eq:ej2_zin_gyr_sin_aprox}. 

De la figura \ref{fig:ej2_zin_gyr_vs_rlc} se observa que tanto en frecuencia como en m\'odulo la impedancia del gyrator aproxima a la de un inductor de $L=0.02$ con $R_L = 50\Omega$ hasta 20KHz.\par
En la figura \ref{fig:ej2_zin_gyr_calc_vs_sim} se observa que la impedancia calculada coincide con la simulada hasta aproximadamente $100KHz$ en fase y $400KHz$ en m\'odulo. A partir de esas frecuencias, la forma de la simulaci\'on es la del c\'alculo pero adelantada. Posiblemente estos efectos se deban a caracter\'isticas no ideales del op-amp como polos secundarios de la ganancia o una impedancia de salida no despreciable. \par

En las figuras siguientes se observa que el m\'odulo de la salida de los filtros con gyrator se separa de la de los filtros con inductor en frecuencias de 20KHz o m\'as altas.


\begin{figure*}
	\includegraphics[width=\textwidth]{imagenes/{"zin_gyr calculo vs sim"}}
	\caption{C\'alculo y simulaci\'on de la impedancia de entrada del gyrator}	
	\label{fig:ej2_zin_gyr_calc_vs_sim}
\end{figure*}

\begin{figure*}
	\includegraphics[width=\textwidth]{imagenes/{"zin_gyr vs zin_rlc"}}
	\caption{Simulaci\'on de la impedancia de entrada del gyrator y de la de una bobina real}
	\label{fig:ej2_zin_gyr_vs_rlc}
\end{figure*}


%\begin{figure*}
%	\includegraphics[width=\textwidth]{imagenes/{"BP_gyr vs BP_rlc"}}
%	\caption{Simulaci\'on filtro band-pass implementado con bobina real y con gyrator}
%	\label{fig:ej2_BP_gyr_vs_BP_rlc}
%\end{figure*}

\begin{figure*}
	\includegraphics[width=\textwidth]{imagenes/{"BR_gyr vs BR_rlc"}}
	\caption{Simulaci\'on filtro band-reject implementado con bobina real y con gyrator}
	\label{fig:ej2_BR_gyr_vs_BR_rlc}
\end{figure*}

\begin{figure*}
	\includegraphics[width=\textwidth]{imagenes/{"HP_gyr vs HP_rlc"}}
	\caption{Simulaci\'on filtro high-pass implementado con bobina real y con gyrator}
	\label{fig:ej2_HP_gyr_vs_HP_rlc}
\end{figure*}

\begin{figure*}
	\includegraphics[width=\textwidth]{imagenes/{"LP_gyr vs LP_rlc"}}
	\caption{Simulaci\'on filtro low-pass implementado con bobina real y con gyrator. Salida del restador}
	\label{fig:ej2_LP_gyr_vs_LP_rlc}
\end{figure*}



\begin{figure*}
	\includegraphics[width=\textwidth]{imagenes/{"LP_gyr no res vs LP_rlc no res"}}
	\caption{Simulaci\'on filtro low-pass implementado con bobina real y con gyrator. Salida diferencial}
	\label{fig:ej2_LP_dif_gyr_vs_rlc}
\end{figure*}































