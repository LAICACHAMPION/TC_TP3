

\documentclass[../../tc_tp3_main.tex]{subfiles}

\begin{document}
%capítulo, Amplificadores de instrumentación
\chapter{Amplificadores de instrumentación}

%Introducción
\section{Introducción}

	%Notación
	A modo de delimitar un marco teórico y notacional a partir del cual se presentarán con mayor claridad y precisión los términos técnicos presentados a continuación, se procede a definir las dos entradas genéricas $V_1$ y $V_2$ de un circuito de tipo MISO (multiple inputs, single output) como: \par
	
	%Notación V_{CM} + V_{DMi}
 	\begin{equation}
  	   \left\{
	  	    \begin{array}{ll}
		 					\mathrm{V_1} = \mathrm{V_{CM} + V_{DM1}} \\
			 				\mathrm{V_2} = \mathrm{V_{CM} + V_{DM2}} \\
	     	 \end{array}
	     	\right.
 	\end{equation}
 	
	donde $V_{CM}$ es la tensión de modo común, es decir, la componente compartida por las dos señales $V_1$ y $V_2$, y $V_{DMi}$ es la tensión diferencial o la componente única/diferente de la tensión i, con i =1;2.\par
	Nótese que tanto $V_{CM}$ como $V_{DMi}$ pueden ser nulas o no dependiendo de las señales $V_1$ y $V_2$ y de la relación existente entre ellas. \par
	En particular, cuando la señal $V_1$ y $V_2$ comparten el mismo canal de transmisión se podrá decir que las señales comparten el ruido proveniente del canal y por lo tanto $V_{CM}$ será una variable aleatoria de distribución de probabilidad acorde a determinar. Además, aquellas señales que estén montadas sobre tensiones continuas también tendrán una componente continua común. \par
	Debe también hacerse notar el hecho de que $V_1 - V_2 = V_{DM1} - V_{DM2}$, por lo que el modo común se verá eliminado al restar las dos señales de input. \par
	Una vez establecidas 
	%definición amplificador diferencial
	Un \underline{amplificador diferencial} es un circuito cuya función será eliminar
	
	%definición CMRR
	El \underline{CMRR} se define como \todo{Explicar CMRR } 
	
	%definición amplificador de instrumentación
	Un amplificador de instrumentación es un amplificador diferencial que cumple con las siguientes condiciones: \par
	\begin{enumerate}
		\item Impedancia de entrada muy grande (idealmente infinita) tanto para el modo diferencial como para el común.
		\item Impedancia de salida muy baja (idealmente nula).
		\item Ganancia estable y precisa.
		\item Un CMRR extremadamente grande.
	\end{enumerate}
	Dado que ... \todo{Explicar la degradación de la CMRR} 
	El 
	
	En este trabajo se justificará el  
\section{Diseño del amplificador de instrumentación}

\end{document}
