\section{Gyrator como simulador de un inductor}

a ciertas frecuencias tiene impedancia de entrada aprox sL+RL, con L =Rgyr RL Cgyr 

\begin{figure*}[htbp]
	\centering
	\begin{circuitikz}[scale=1]
%\draw[help lines]   			grid(6,6);	
\draw								(0,3)
		to [R=$R_L$,i>^=$i_{in}$]	(3,3)
		to [L=$L{=}R_LRC$]			(3,0) node[ground]{};
\end{circuitikz}

	\begin{circuitikz}[scale=1]
\draw[help lines]   			grid(6,6);	

\draw									(0,3)
		to [short, o-*, i>^=$i_{in}$]	(1,3)
		to [short, i>_=$i_A$]			(1,4)
		to [R=$R_L$]				(3,4)
		to [short] 					(3,5)
		to [short]					(5.5,5)
		to [short]					(5.5,3.5)

									(3,2)
		to [short]					(3,3)
									(1,3)
		to [short, i>^=$i_B$]		(1,2)
		to [C=$C_1$]				(3,2)
		to [R=$R_1$]				(3,0) node[ground]{}
		
									(4.25,3.5)
		node[op amp](opamp){};
\end{circuitikz}

	\caption{Uso de gyrator como inductor}
	\label{fig:ej2_gyrator_simulacion_inductor}
\end{figure*}




En la secci\'on \ref{ssec:ej2_obtencion_zin_gyr} se obtiene que la impedancia de entrada del gyrator es

\begin{equation}
	Z_{in}= \frac{ sC_{gyr}R_{gyr}R_L+R_L}{sC_{gyr}(R_{gyr}(1-K)+R_L) + 1} \label{eq:ej2_zin_gyr_sin_aprox} \\
\end{equation}
	en donde	$K  \approx\frac{1}{1+\frac{s}{2\pi BWP}}$ (ecuaci\'on \ref{eq:ej2_K}).
 
Si el denominador puede aproximarse a 1 (ver secci\'on \ref{ssec:ej2_zin_gyr_aprox_inductor}), la impedancia de entrada se puede aproximar a la del modelo de un inductor con resistencia serie con valores $L=CR_LR_1$ y $R_{coil}=R_L$

\begin{align}
	Z_{in} &= sC_{gyr}R_LR_{gyr}+R_L \\
 	\abs{Z_{in}} &= R_L \, \sqrt{4\pi^2f^2C_{gyr}^2R_{gyr}^2+1} \\
 	\phase{Z_{in}} &= arctg(2\pi fC_{gyr}R_{gyr})
	\label{eq:ej2_zin_gyrator_con_aprox}
\end{align}



\subsection{Limitaciones en frecuencia.} \label{ssec:ej2_zin_gyr_aprox_inductor}

%condiciones para la aproximacion
Se analiza el denominador de la impedancia de entrada (ecuaci\'on \ref{eq:ej2_zin_gyr_sin_aprox}) y se busca en qu\'e rango de frecuencias puede aproximarse a 1. Siendo $D(s)$ el denominador:

\[D(s) = sC_{gyr}(R_{gyr}(1-\frac{1}{1+\frac{s}{2\pi BWP}})+R_L) + 1 \]
\[	= \frac{C_{gyr}R_T\, s^2 + 2\pi BWP\, C_{gyr} R_L \, s + 2\pi BWP}{s+2\pi BWP}   \]
Siendo $R_T = R_L + R_{gyr}$.
\[=\frac{\frac{C_{gyr}R_T}{2\pi BWP}s^2 + C_{gyr}R_L s +1}{\frac{s}{2\pi BWP}+1}\]

A continuaci\'on se analizan los polos y ceros de $D(s)$:

\begin{description}

	\item[\textbf{Cero doble en $f_{cero} = \frac{1}{2\pi}\sqrt{\frac{2\pi BWP}{C_{gyr}R_T}}$: }]



Se observa que el denominador tiene un cero doble en $f_0 = \frac{1}{2\pi}\sqrt{\frac{2\pi BWP}{C_{gyr}R_T}}$. La existencia o no del sobrepico (y de existir, su altura y posici\'on) depende de la relaci\'on entre el coeficiente lineal $b = C_{gyr}R_L$ y $f_0$. 

\[b = \frac{2\xi}{2\pi f_0}\]
\[\xi = \frac{2\pi f_0 b}{2}\]
\[\xi = \frac{\sqrt{\frac{2\pi BWP}{C_{gyr}R_T}} \cdot C_{gyr}R_L}{2}   \]
\[ = \frac{1}{2}R_L\sqrt{\frac{2\pi BWP C_{gyr}}{R_L + R_{gyr}}}\]
Si $R_{gyr} \gg R_L$:
\[ \approx \frac{1}{2}R_L\sqrt{\frac{2\pi BWP C_{gyr}}{R_{gyr}}}\]
Si $\xi \leqslant \frac{1}{\sqrt{2}}$, $D(s)$ tiene sobrepico (va a ser cada vez mas pronunciado mientras m\'as chico sea $\xi$). 


	\item[\textbf{Polo simple en $f_{polo} = BWP$}]
	
	NO ESTA EN BWP MITERIO MITERIO MITERIO MITERIO


\end{description}

$f_{cero} < f_{polo} \Rightarrow f_{cero}$ es la frecuencia que me delimita hasta donde $D(s)\approx 1$, y en consecuencia hasta donde anda el gyrator. EN VERDAD ANDA HASTA MENOS DE D(S) PORQUE HAY SOBREPICO Y TODA ESA GILADA.



\subsection{Eleccion de componentes}
\begin{align*}
	R_L &= 50 \Omega \\
	R_{gyr} &= 39k\Omega \\
	C_{gyr} &= 10nF
\end{align*}



\begin{figure*}
	\includegraphics[width=\textwidth]{imagenes/{"zin_gyr calculo vs sim"}}
	\label{fig:ej2_zin_gyr_calc_vs_sim}
	\caption{C\'alculo y simulaci\'on de la impedancia de entrada del gyrator}
\end{figure*}

\begin{figure*}
	\includegraphics[width=\textwidth]{imagenes/{"zin_gyr vs zin_rlc"}}
	\label{fig:ej2_zin_gyr_vs_rlc}
	\caption{Simulaci\'on de la impedancia de entrada del gyrator y de la de una bobina real}
\end{figure*}





\begin{figure*}
	\includegraphics[width=\textwidth]{imagenes/{"BP_gyr vs BP_rlc"}}
	\label{fig:ej2_BP_gyr_vs_BP_rlc}
	\caption{Simulaci\'on filtro band-pass implementado con bobina real y con gyrator}
\end{figure*}

\begin{figure*}
	\includegraphics[width=\textwidth]{imagenes/{"BR_gyr vs BR_rlc"}}
	\label{fig:ej2_BR_gyr_vs_BR_rlc}
	\caption{Simulaci\'on filtro band-reject implementado con bobina real y con gyrator}
\end{figure*}

\begin{figure*}
	\includegraphics[width=\textwidth]{imagenes/{"HP_gyr vs HP_rlc"}}
	\label{fig:ej2_HP_gyr_vs_HP_rlc}
	\caption{Simulaci\'on filtro high-pass implementado con bobina real y con gyrator}
\end{figure*}

\begin{figure*}
	\includegraphics[width=\textwidth]{imagenes/{"LP_gyr vs LP_rlc"}}
	\label{fig:ej2_LP_gyr_vs_LP_rlc}
	\caption{Simulaci\'on filtro low-pass implementado con bobina real y con gyrator. Salida del restador}
\end{figure*}



\begin{figure*}
	\includegraphics[width=\textwidth]{imagenes/{"LP_gyr no res vs LP_rlc no res"}}
	\label{fig:ej2_LP_dif_gyr_vs_rlc}
	\caption{Simulaci\'on filtro low-pass implementado con bobina real y con gyrator. Salida diferencial}
\end{figure*}








\begin{figure*}[htbp]
\missingfigure{pzplot del denominador de zgyr con los valores finales}
\label{fig:ej2_pzplot_den_zgyr}
\caption{Diagrama de polos y ceros de la impedancia de entrada del gyrator}
\end{figure*}

































\subsection{Otras limitaciones}

\begin{description}	
	\item[Alamcenamiento energ\'etico] no puede almacenar energ\'ia de la misma manera que un inductor. La magnitud de la fem producida ante cambios de corriente $\left( V = \frac{di}{dt} \right) $ tiene limitaciones propias de las caracter\'isticas el\'ectricas del circuito (ej.: op-amp no puede largar 100.000kV a pesar de lo que diga spice) 
	\item[Terminal a tierra] una de las terminales del inductor simulado siempre debe estar a tierra \todo{no entiendo cmo se relaciona esto con el modelo de gyrator como cuadripolo que dio dani}
	
	\item[Propiedades magn\'eticas] No crean campos magn\'eticos de la misma forma que los inductores, por lo que no se puede conseguir un efecto de mutua inducci\'on. \footnote{por eso no se puede hacer un transformador con desacople el\'ectrico como si se puede hacer con bobinas posta. Si se puede hacer un transformador poniendo dos en cascada pero es no tiene nada que ver y no hay desacople el\'ectrico. Desacople el\'ectrico es un t\'ermino que existe o o invente?}
	
	\item[Transformador de aislaci\'on] un transformador implementado con gyrators no tiene aislaci\'on el\'ectrica como si tiene un transformador real. Por ejemplo, no se podr\'ia implementar un transformador de aislaci\'on \todo{traduccion de isolation transformer esta bien?}

\end{description}


