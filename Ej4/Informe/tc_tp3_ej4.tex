

\documentclass[../../tc_tp3_main.tex]{subfiles}

\begin{document}
\chapter{Control de tonos y ecualizador de fase}
\todo{introduccion al tema}

\begin{figure}[H]
\centering
\includegraphics[width=0.3\textwidth]{imagenes/circuitoControlTonos.png}
\caption{Circuito del control de tonos} \label{fig:cct}
\end{figure}




\section{Función transferencia}

Para hallar la función transferencia $H(S)=\frac{V_{o}}{V_{in}} $del circuito de la figura \ref{fig:cct}, se realizaron transformaciones estrella a triangulo y viceversa, reduciendo el circuito. Dichas transformaciones se realizaron con Matlab.



\begin{figure}[H]
\centering
\includegraphics[width=0.5\textwidth]{imagenes/simpl1.png}
\caption{Transformación triangulo a estrella} \label{fig:cs1}
\end{figure}

Reemplazando el circuito triangulo por el estrella, permitio agrupar $A$ con $R_1$, $C$ con$ R_1$ y$ B$ con $C_1$. Obteniendo un nuevo circuito estrella con las siguientes impedancias $ K_1$ , $K_2$ y  $K_3$, tal como se observa en la figura \ref{fig:cs2}.

\begin{gather}
   A=\frac{K\, R_{2}}{C_{1}\, S\, \left(\frac{1}{C_{1}\, S} - R_{2}\, \left(K - 1\right) + K\, R_{2}\right)} \\
B=-\frac{R_{2}\, \left(K - 1\right)}{C_{1}\, S\, \left(\frac{1}{C_{1}\, S} - R_{2}\, \left(K - 1\right) + K\, R_{2}\right)}\\
C=-\frac{K\, {R_{2}}^2\, \left(K - 1\right)}{\frac{1}{C_{1}\, S} - R_{2}\, \left(K - 1\right) + K\, R_{2}}
\end{gather}


\begin{figure}[H]
\centering
\includegraphics[width=0.5\textwidth]{imagenes/simpl2.png}
\caption{Agrupo impedancias en serie} \label{fig:cs2}
\end{figure}


\begin{figure}[H]
\centering
\includegraphics[width=0.5\textwidth]{imagenes/simpl3.png}
\caption{Transformación estrella a triangulo} \label{fig:cs3}
\end{figure}

Dicho circuito estrella se lo transformó a triangulo para de esta manera poder agrupar $F_2$ con $R_3$ y $F_3$ con $R_3$ (figura \ref{fig:cs4}).




\begin{figure}[H]
\centering
\includegraphics[width=0.5\textwidth]{imagenes/simpl4.png}
\caption{Agrupo impedancias en paralelo} \label{fig:cs4}
\end{figure}



\begin{figure}[H]
\centering
\includegraphics[width=0.5\textwidth]{imagenes/simplFin.png}
\caption{Circuito equivalente} \label{fig:csFin}
\end{figure}

Finalmente se obtiene el circuito de la figura \ref{fig:csFin}. Considerando que el OpAmp se comporta idealmente y la corriente que circula internamente por la entrada del amplificador es cero, resulta la siguiente  función transferencia:

\begin{equation}
H(S)=- \frac{X_2}{X_1}\label{eq:circuitoRed}
\end{equation}

Donde
\todo{ver que pasa con las ecuaciones} 
\begin{gather}
X_1=    \\
X_2=
\end{gather}

Aplicando las siguientes condiciones de diseño sobre la función transferencia
\begin{gather}
 R_3 >> R_1   \\
 R_3 =10 R_2   \\
C_1=10C_2
\end{gather}

Obtenemos

\todo{ecuacion reducida}

Si
\begin{equation}
\begin{split}
-20C_{2}^2 K^2 R_{2}^2 R_{1}  S^{2} +    20 C_{2}^{2} K R_{1} R_{2}^2+ 10 C_{2}^2 R_{1}^2 R_{2} S^{2} + 100 C_{2}^{2} R_{1} R_{2}^2 S^{2}  \\   \approx 100 C_{2}^2 R_{1} R_{2} ^2 S^2
\end{split}
\end{equation}



\todo{ecuacion reducida} 

La ecuacion xx posee la forma

\begin{equation}
H(S)=\frac {\left( \frac{S}{W_0} \right) ^2 + \frac{S}{Q_Z W_0} +1}{\left( \frac{S}{W_0} \right) ^2 +\frac{S}{Q_Z W_0}+1} \label{eq:MpasaBanda}
\end{equation}
Dicha función transferencia corresponde a un circuito pasa banda de segundo orden, donde $W_0$ es la frecuencia central de la banda y $Q_Z$ , $Q_P$, son los respectivos factores de calidad.
\subsection{Frecuencia central}
El coeficiente normalizado del término $S^2$, tanto para el numerado como en el denominador (ecuación zz), es:
\begin{equation}
\frac{100 C_{2}^2 R_{1} R_{2} ^2}{2R_1+R_2}  \label{eq:coefP}
\end{equation}
Entonces por la ecuación \ref{eq:MpasaBanda} y \ref{eq:coefP}, se obtiene la frecuencias central del pasa banda.

\begin{gather}
W_0^2=\frac{1}{ecuacion \ref{eq:coefP}}=\frac{2R_1+R_2}{100 C_{2}^2 R_{1} R_{2} ^2}   \\
W_0=\frac{\sqrt{2+\frac{R_2}{R_1}}}{10C_2 R_2}\\
f_0=\frac{\sqrt{2+\frac{R_2}{R_1}}}{20 \pi C_2 R_2}
\end{gather}
\subsection{Factores de calidad}
A partir de los coeficientes normalizados de los términos de S, obtenemos los factores de calidad




\subsubsection{Factor de calidad $Q_Z$}
El coeficiente normalizado correspondiente al termino de $S$ del numerador, es
\begin{equation}
\frac{-C_2 K ^2 R_2^2 - 9  C_2 K R_2 ^2 + C_2 R_1^2 +31 C_2 R_1 R_2+10 C_2 R_2^2}{2R_1 +R_2} \label{eq:coefNUM}
\end{equation}


Entonces por la ecuación \ref{eq:MpasaBanda} y \ref{eq:coefNUM}, se obtiene:

\begin{gather}
Q_Z=\frac{1}{W_0 \cdot ecuacion\ref{eq:coefNUM}}\\
Q_Z=\frac{\left( 2R_1 +R_2 \right) 10 R_2 }{\left( - K ^2 R_2^2 - 9   K R_2 ^2 +  R_1^2 +31  R_1 R_2+10  R_2^2 \right)\sqrt{2+\frac{R_2}{R_1}} }
\end{gather}







\subsubsection{Factor de calidad $Q_P$}
El coeficiente normalizado correspondiente al termino de $S$ del denominador, es
\begin{equation}
\frac{-C_2 K ^2 R_2^2 + 11  C_2 K R_2 ^2 + C_2 R_1^2 +31 C_2 R_1 R_2}{2R_1 +R_2} \label{eq:coefDEN}
\end{equation}


Entonces por la ecuación \ref{eq:MpasaBanda} y \ref{eq:coefDEN}, se obtiene:

\begin{gather}
Q_P=\frac{1}{W_0 \cdot ecuacion\ref{eq:coefDEN}}\\
Q_P=\frac{\left( 2R_1 +R_2 \right) 10 R_2 }{\left( - K ^2 R_2^2 + 11   K R_2 ^2 +  R_1^2 +31  R_1 R_2 \right)\sqrt{2+\frac{R_2}{R_1}} }
\end{gather}


\subsubsection{Modulo de $H(f)$ en $W_0$}
Definimos $A_0$ como:
\begin{equation}
A_0	\widehat{=} \mid H(S=jW_0) \mid
\end{equation}

Reemplazamos $S=jW_0$ en \ref{eq:MpasaBanda}

\begin{gather}
A_0 =\frac{Q_P}{Q_Z}\\
A_0=\frac{- K ^2 R_2^2 - 9   K R_2 ^2 +  R_1^2 +31  R_1 R_2+10  R_2^2}{ - K ^2 R_2^2 + 11   K R_2 ^2 +  R_1^2 +31  R_1 R_2 }
\end{gather}
Si $K=0$

\begin{equation}
\begin{split}
A_0(K=0) =\frac{R_1^2 +31  R_1 R_2+10  R_2^2}{  R_1^2 +31  R_1 R_2}\\
= \frac{R_1(R_1+31R_2)+10R_2^2}{R_1(R_1+31R_2)}\\
\approx \frac{R_1 31 + 10 R_2}{31 R_1}
\end{split}
\end{equation}
\begin{equation}
A_0(K=0) \approx \frac {3R_1 + R_2}{3R_1} \label{eq:A00}
\end{equation}

Si $K=1$

\begin{equation}
\begin{split}
A_0(K=1) =\frac{R_1^2 + 31R_1 R_2}{10 R_2^2 +R_1^2+31R_1 R_2} \\
\approx \frac{R_1 31}{10 R_2 + 31 R_1} \\
 \approx \frac{3 R_1}{R_2 + 3 R_1}
\end{split}
\end{equation}

\begin{equation}
A_0(K=1) \approx \frac {3R_1}{3R_1 + R_2}\label{eq:A01}
\end{equation}

A partir de \ref{eq:A00} y \ref{eq:A01} obtenemos que

\begin{equation}
\frac {3R_1}{3R_1 + R_2} \leq A_0  \leq \frac{3R_1 + R_2}{3R_1}
\end{equation}














\end{document}
