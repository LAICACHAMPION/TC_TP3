\section{Gyrator}

\subsection{Uso como simulador de un inductor. Limitaciones en frecuencia.}

\begin{figure*}[htbp]
	\centering
	\begin{circuitikz}[scale=1]
%\draw[help lines]   			grid(6,6);	
\draw								(0,3)
		to [R=$R_L$,i>^=$i_{in}$]	(3,3)
		to [L=$L{=}R_LRC$]			(3,0) node[ground]{};
\end{circuitikz}

	\begin{circuitikz}[scale=1]
\draw[help lines]   			grid(6,6);	

\draw									(0,3)
		to [short, o-*, i>^=$i_{in}$]	(1,3)
		to [short, i>_=$i_A$]			(1,4)
		to [R=$R_L$]				(3,4)
		to [short] 					(3,5)
		to [short]					(5.5,5)
		to [short]					(5.5,3.5)

									(3,2)
		to [short]					(3,3)
									(1,3)
		to [short, i>^=$i_B$]		(1,2)
		to [C=$C_1$]				(3,2)
		to [R=$R_1$]				(3,0) node[ground]{}
		
									(4.25,3.5)
		node[op amp](opamp){};
\end{circuitikz}

	\caption{Uso de gyrator como inductor}
	\label{fig:ej2_gyrator_simulacion_inductor}
\end{figure*}


\subsubsection{Obtenci\'on impedancia de entrada $Z_{in}$}

Para el siguiente c\'alculo se desprecian las corrientes de bias y la tensi\'on de offset.\\

Relaci\'on entre $V^-$ y $V^+$:
\begin{align}
V^- &= A_{vol}\left( V^+ - V^-  \right)	\\
V^- \left( 1 + A_{vol}\right) &= A_{vol}\, V^+ \\
V^- &= V^+\frac{A_{vol}}{1+A_{vol}}\\
V^- &= V^+K \label{eq:ej2_relacion_entradas_opamp_gyrator}
\end{align}

Con $K=\frac{A_{vol}}{1+A_{vol}}$.
Usando el modelo de $A_{vol}$ del polo dominante se obtiene la expresion de K:

%relaicion entre V- y V+
\begin{align}
K&= \frac{\frac{A_o}{\frac{s}{\omega_p}+1}}{\frac{A_o}{\frac{s}{\omega_p}+1}+1} \\
 &= \frac{A_o}{(A_o + 1) + \frac{s}{\omega_p}}\\
 &= \frac{A_o}{A_o+1} \cdot \frac{1}{1+\frac{s}{(A_o+1)\omega_p}}\\
 \intertext{Considerando que $A_o+1\approx A_o$:}
  &=\frac{1}{1+\frac{s}{BWP}}
 \intertext{Siendo $BWP=A_o\cdot \omega_p$}
\end{align}
\todo{BWP es con 2pi o sin?}

Se buscan las tensiones en las entradas del \textit{op-amp} para luego hallar las corrientes $i_A$ y $i_B$.


%relacion de V- y V+ con Vin
\begin{align}
	\intertext{Por divisor resistivo:}
	V^+&=V_{in}\frac{R_1}{R_1+\frac{1}{sC}} \\
	\intertext{De la ecuaci\'on \ref{eq:ej2_relacion_entradas_opamp_gyrator}:}
	V^-&=V_{in}\cdot K \frac{R_1}{R_1+\frac{1}{sC}}
\end{align} 

%i_A
\begin{align}
i_A &= \frac{1}{R_L}\left(V_{in} - V^-\right)\\
 &= V_{in} \frac{1}{R_L}\left( 1-K\frac{R_1}{R_1+\frac{1}{sC}} \right)\\
 &= V_{in} \frac{sCR_1+1-KsCR_1}{R_L\left( sCR_1+1 \right)} \\
 &= V_{in} \frac{(1-K)sCR_1+1}{R_L\left( sCR_1+1 \right)}
\end{align}


%i_B
\begin{align}
i_B &= V_{in} \frac{1}{\frac{1}{sC} + R_1}\\
	&= V_{in} \frac{sC}{1+sCR_1}
\end{align}

%i_{in}
\begin{align}
i_{in} &= i_A + i_B \\
 &= V_{in} \left( \frac{(1-K)sCR_1+1}{R_L\left( sCR_1+1 \right)} +   \frac{sC}{1+sCR_1}   \right) \\ 
 &= V_{in}  \frac{(1-K)sCR_1+1 + sCR_L}{R_L\left( sCR_1+1 \right)} \\
 &= V_{in}  \frac{sC(R_1(1-K)+R_L) + 1}{R_L(sCR_1+1)} 
\end{align}

De este resultado se obtiene la impedancia de entrada:

\begin{equation}
Z_{in}= \frac{ sCR_1R_L+R_L}{sC(R_1(1-K)+R_L) + 1}
\end{equation}


%condiciones para la aproximacion
\begin{description}
	\item [\textbf{$K \approx 1$}] Elijo criterio: K tiene la forma de una transferencia de un filtro pasabajos de primer orden. Siendo $f_0$ la frecuencia de corte, se considera que $K\approx 1$ si $f<\frac{f_0}{10}$, es decir, una d\'ecada antes de la frecuencia de corte. En este caso, $f_0=BWP$ y la aproximaci\'on es v\'alida para $f<\frac{BWP}{10}$
	
	\item [\textbf{$sCR_L +1 \approx 1$}] Elijo criterio: $2\pi fCR_L <0.05 \iff f< \frac{0.05}{2\pi CR_L}$
\end{description}

La restricci\'on de BWP es independiente de la eleccion de componentes, osea que es fija. busco con la eleccion de componentes que la otra restriccion sea mas laxa que la primera para que funcione durante mas frecuencia.

Usando TL082 con $BWP=8MHz$:
\begin{align*}
\frac{BWP}{10} \leqslant \frac{0.05}{2\pi C R_L} \\
\Rightarrow CR_L\leqslant \frac{1}{4\pi BWP} \approx 20ns
\end{align*}




Si la frecuencia cumple con las dos condiciones anteriores, la impedancia de entrada se puede aproximar a la del modelo de un inductor con resistencia serie con valores $L=CR_LR_1$ y $R_{coil}=R_L$

\begin{align}
	Z_{in} &= sCR_LR+R_L \\
 	\abs{Z_{in}} &= R_L \, \sqrt{4\pi^2f^2C^2R^2+1} \\
 	\phase{Z_{in}} &= arctg(2\pi fCR)
	\label{eq:ej2_zin_gyrator_con_aprox}
\end{align}

Observar que $R_1$ no tiene restricciones sobre qu\'e valores puede tomar para que el gyrator se comporte como un inductor, no afecta a la resistencia serie final, y si afecta a la L, osea que es el valor clave para modificar 

 

\subsection{Criterios de dise\~no}

\begin{enumerate}
	\item Elijo $CR_L < \frac{1}{4\pi BWP}$
	\item Elijo $R_L < \frac{Rcircuito}{20}$. De ahi obtengo $C$
	\item Elijo $Rgyrator = \frac{L}{CR_L}$
\end{enumerate}






\subsection{Otras limitaciones}

\begin{description}
	\item[Funcionamiento a altas y bajas frecuencias]
	\todo[inline]
	
	\item[Alamcenamiento energ\'etico] no puede almacenar energ\'ia de la misma manera que un inductor. La magnitud de la fem producida ante cambios de corriente $\left( V = \frac{di}{dt} \right) $ tiene limitaciones propias de las caracter\'isticas el\'ectricas del circuito (ej.: op-amp no puede largar 100.000kV a pesar de lo que diga spice) 
	\item[Terminal a tierra] una de las terminales del inductor simulado siempre debe estar a tierra \todo{no entiendo cmo se relaciona esto con el modelo de gyrator como cuadripolo que dio dani}
	
	\item[Propiedades magn\'eticas] No crean campos magn\'eticos de la misma forma que los inductores, por lo que no se puede conseguir un efecto de mutua inducci\'on. \footnote{por eso no se puede hacer un transformador con desacople el\'ectrico como si se puede hacer con bobinas posta. Si se puede hacer un transformador poniendo dos en cascada pero es no tiene nada que ver y no hay desacople el\'ectrico. Desacople el\'ectrico es un t\'ermino que existe o o invente?}
	
	\item[ddddd] un transformador implementado con gyrators no tiene aislaci\'on el\'ectrica como si tiene un transformador real. Por ejemplo, no se podr\'ia implementar un transformador de aislaci\'on \todo{traduccion de isolation transformer esta bien?}

\end{description}


