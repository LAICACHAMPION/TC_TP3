\newpage
\section{Anexo}

\subsection{Obtenci\'on funciones transferencia RLC ideal}
\label{ssec:ej2_H_rlc_ideal}

Para todos los filtros se considera 
\begin{align*}
\omega_0 &= \frac{1}{\sqrt{LC}}\\
\alpha &=\frac{R}{2L}\\
\xi &= \frac{\alpha}{\omega_0} \\ &= \frac{R}{2}\sqrt{\frac{C}{L}} \\
Q&=\frac{1}{2\xi} \\ &= \frac{1}{R}\sqrt{\frac{C}{L}}
\end{align*}


\todo{repetir todas las H pero con RL en serie con la bobina}

\subsubsection{Filtro high-pass}

RLC serie tomando la salida en el capacitor:

\begin{align}
H(s)&=\frac{\frac{1}{sC}}{\frac{1}{sC}+R+sL}	\\
&= \frac{1}{s^2 \, LC + s\, RC + 1}	\\
&= \frac{1}{\left(\frac{s}{\omega_0}\right)^2 + \frac{s}{\omega_0}\, 2\xi + 1}
\end{align}


\subsubsection{Filtro low-pass}

$RLC$ serie tomando la salida en el inductor:

\begin{align}
H(s)&=\frac{sL}{\frac{1}{sC}+R+sL}	\\
&= \frac{s^2LC}{s^2 \, LC + s\, RC + 1}	\\
&= \frac{\left(\frac{s}{\omega_0}\right)^2}{\left(\frac{s}{\omega_0}\right)^2 + \frac{s}{\omega_0}\, 2\xi + 1}
\end{align}

\subsubsection{Filtro band-reject}

$RLC$ serie tomando la salida en el inductor y el capacitor:

\begin{align}
H(s)&=\frac{sL + \frac{1}{sC}}{\frac{1}{sC}+R+sL}	\\
&= \frac{s^2LC+1}{s^2 \, LC + s\, RC + 1}	\\
&= \frac{\left(\frac{s}{\omega_0}\right)^2+1}{\left(\frac{s}{\omega_0}\right)^2 + \frac{s}{\omega_0}\, 2\xi + 1}
\end{align}

\subsubsection{Filtro band-pass}

$R$ en serie con un paralelo $LC$ tomando la salida en el paralelo:

\todo[inline]{H BP}

\subsection{Obtenci\'on impedancia de entrada $Z_{in}$} \label{ssec:ej2_obtencion_zin_gyr}

Para el siguiente c\'alculo se desprecian las corrientes de bias y la tensi\'on de offset.\\

Relaci\'on entre $V^-$ y $V^+$:
\begin{align}
V^- &= A_{vol}\left( V^+ - V^-  \right)	\\
V^- \left( 1 + A_{vol}\right) &= A_{vol}\, V^+ \\
V^- &= V^+\frac{A_{vol}}{1+A_{vol}}\\
V^- &= V^+K \label{eq:ej2_relacion_entradas_opamp_gyrator}
\end{align}

Con $K=\frac{A_{vol}}{1+A_{vol}}$.
Usando el modelo de $A_{vol}$ del polo dominante se obtiene la expresion de K:

%relacion entre V- y V+
\begin{align}
K&= \frac{\frac{A_o}{\frac{s}{\omega_p}+1}}{\frac{A_o}{\frac{s}{\omega_p}+1}+1} \\
 &= \frac{A_o}{(A_o + 1) + \frac{s}{\omega_p}}\\
 &= \frac{A_o}{A_o+1} \cdot \frac{1}{1+\frac{s}{(A_o+1)\omega_p}}\\
 \intertext{Considerando que $A_o+1\approx A_o$:}
  K&=\frac{1}{1+\frac{s}{2\pi BWP}}	\label{eq:ej2_K}
 \intertext{Siendo $2\pi BWP=A_o\cdot \omega_p$}
\end{align}


Se buscan las tensiones en las entradas del \textit{op-amp} para luego hallar las corrientes $i_A$ y $i_B$.

%relacion de V- y V+ con Vin
\begin{align}
	\intertext{Por divisor resistivo:}
	V^+&=V_{in}\frac{R_{gyr}}{R_{gyr}+\frac{1}{sC_{gyr}}} \\
	\intertext{De la ecuaci\'on \ref{eq:ej2_relacion_entradas_opamp_gyrator}:}
	V^-&=V_{in}\cdot K \frac{R_{gyr}}{R_{gyr}+\frac{1}{sC_{gyr}}}
\end{align} 

%i_A, i_B, i_A+i_B
\begin{align*}
i_A &= \frac{1}{R_L}\left(V_{in} - V^-\right)\\
 &= V_{in} \frac{1}{R_L}\left( 1-K\frac{R_{gyr}}{R_{gyr}+\frac{1}{sC_{gyr}}} \right)\\
 &= V_{in} \frac{sCR_{gyr}+1-KsC_{gyr}R_{gyr}}{R_L\left( sC_{gyr}R_{gyr}+1 \right)} \\
 &= V_{in} \frac{(1-K)sC_{gyr}R_{gyr}+1}{R_L\left( sC_{gyr}R_{gyr}+1 \right)}\\
 & \\
i_B &= V_{in} \frac{1}{\frac{1}{sC_{gyr}} + R_{gyr}}\\
	&= V_{in} \frac{sC_{gyr}}{1+sC_{gyr}R_{gyr}}\\
 & \\
i_{in} &= i_A + i_B \\
 &= V_{in} \left( \frac{(1-K)sC_{gyr}R_{gyr}+1}{R_L\left( sC_{gyr}R_{gyr}+1 \right)} +   \frac{sC_{gyr}}{1+sC_{gyr}R_{gyr}}   \right) \\ 
 &= V_{in}  \frac{(1-K)sC_{gyr}R_{gyr}+1 + sC_{gyr}R_L}{R_L\left( sC_{gyr}R_{gyr}+1 \right)} \\
 &= V_{in}  \frac{sC_{gyr}(R_{gyr}(1-K)+R_L) + 1}{R_L(sC_{gyr}R_{gyr}+1)}
\end{align*}

De este resultado \ref{eq:ej2_iin_gyr} se obtiene la impedancia de entrada:

\begin{equation}
	Z_{in}= \frac{ sCR_{gyr}R_L+R_L}{sC_{gyr}(R_{gyr}(1-K)+R_L) + 1} \label{eq:ej2_zin_gyr_sin_aprox} \\
\end{equation}
	en donde	$K  \approx\frac{1}{1+\frac{s}{2\pi BWP}}$ (ecuaci\'on \ref{eq:ej2_K}).


impedancia de entrada gyrator

como sale la media logaritmica
